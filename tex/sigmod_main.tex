\documentclass[sigconf]{acmart}



%%
%% \BibTeX command to typeset BibTeX logo in the docs
\AtBeginDocument{%
  \providecommand\BibTeX{{%
    Bib\TeX}}}

%% Rights management information.  This information is sent to you
%% when you complete the rights form.  These commands have SAMPLE
%% values in them; it is your responsibility as an author to replace
%% the commands and values with those provided to you when you
%% complete the rights form.
\setcopyright{acmcopyright}
\copyrightyear{2018}
\acmYear{2018}
\acmDOI{XXXXXXX.XXXXXXX}

%% These commands are for a PROCEEDINGS abstract or paper.
\acmConference[Conference acronym 'XX]{Make sure to enter the correct
  conference title from your rights confirmation emai}{June 03--05,
  2018}{Woodstock, NY}
%%
%%  Uncomment \acmBooktitle if the title of the proceedings is different
%%  from ``Proceedings of ...''!
%%
%%\acmBooktitle{Woodstock '18: ACM Symposium on Neural Gaze Detection,
%%  June 03--05, 2018, Woodstock, NY}
\acmPrice{15.00}
\acmISBN{978-1-4503-XXXX-X/18/06}



% Recommended, but optional, packages for figures and better typesetting:
\usepackage{microtype}
\usepackage{graphicx}
\usepackage{tabularx}
\usepackage{subfigure}
\usepackage{booktabs} % for professional tables
\usepackage{makecell}
\usepackage{caption}
\usepackage{hyperref}


% Attempt to make hyperref and algorithmic work together better:
\newcommand{\theHalgorithm}{\arabic{algorithm}}

% For theorems and such
\usepackage{amsmath}  \usepackage{mathtools} \usepackage{amsthm}

% if you use cleveref..
\usepackage[capitalize,noabbrev]{cleveref}

%%%%%%%%%%%%%%%%%%%%%%%%%%%%%%%%
% THEOREMS
%%%%%%%%%%%%%%%%%%%%%%%%%%%%%%%%
\theoremstyle{plain}
\newtheorem{theorem}{Theorem}[section]
\newtheorem{proposition}[theorem]{Proposition}
\newtheorem{lemma}[theorem]{Lemma}
\newtheorem{question}[theorem]{Question}
\newtheorem{corollary}[theorem]{Corollary}
\newtheorem{fact}[theorem]{Fact}
\theoremstyle{definition}
\newtheorem{definition}[theorem]{Definition}
\newtheorem{assumption}[theorem]{Assumption}
\theoremstyle{remark}
\newtheorem{remark}[theorem]{Remark}

% Todonotes is useful during development; simply uncomment the next line
%    and comment out the line below the next line to turn off comments
%\usepackage[disable,textsize=tiny]{todonotes}
\usepackage[textsize=tiny]{todonotes}




\DeclareMathOperator{\polylog}{polylog}
\DeclareMathOperator{\cost}{cost}
\DeclareMathOperator{\dist}{dist}
\DeclareMathOperator{\poly}{poly}


\newcommand{\R}{\mathbb{R}}
\newcommand{\E}{\mathbb{E}}
\newcommand{\coreset}{\Omega}
\newcommand{\calC}{\mathcal C}
\newcommand{\calD}{\mathcal D}
\newcommand{\calS}{\mathcal S}
\newcommand{\calA}{\mathcal A}
\newcommand{\opt}{\text{OPT}}
\newcommand{\eps}{\varepsilon}

\newcommand{\fkmeans}{\textsc{Fast-kmeans++}~}

\newcommand{\lpar}{\left(}
\newcommand{\rpar}{\right)}
\newcommand{\lbra}{\left\{}
\newcommand{\rbra}{\right\}}
\newcommand{\lnor}{\left\|}
\newcommand{\rnor}{\right\|}


\newcounter{sideremark}
\newcommand{\marrow}{\stepcounter{sideremark}\marginpar{$\boldsymbol{
\longleftarrow\scriptstyle\arabic{sideremark}}$}}

 \newcommand{\david}[1]{
 %  \ifdraft{
    \textsf{{\color{blue} *** (David) \marrow #1 ***}}
 %  }
 %  \fi
 }
  \newcommand{\andrew}[1]{
 %  \ifdraft{
    \textsf{{\color{red} *** (Andrew) \marrow #1 ***}}
 %  }
 %  \fi
 }
 
   \newcommand{\chris}[1]{
 %  \ifdraft{
    \textsf{{\color{green} *** (Chris) \marrow #1 ***}}
 %  }
 %  \fi
 }

 
%for having enumerates with (i) (ii) (iii): 
\renewcommand{\labelenumi}{\theenumi}
 

\usepackage[utf8]{inputenc} % allow utf-8 input
\usepackage[T1]{fontenc}    % use 8-bit T1 fonts
\usepackage{hyperref}       % hyperlinks
\usepackage{url}            % simple URL typesetting
\usepackage{booktabs}       % professional-quality tables
\usepackage{amsfonts}       % blackboard math symbols
\usepackage{nicefrac}       % compact symbols for 1/2, etc.
\usepackage{microtype}      % microtypography
\usepackage{xcolor}         % colors
\usepackage{algorithm}
\usepackage[noend]{algpseudocode}


\title{Settling Time vs. Accuracy Tradeoffs for Clustering Big Data}


% The \author macro works with any number of authors. There are two commands
% used to separate the names and addresses of multiple authors: \And and \AND.
%
% Using \And between authors leaves it to LaTeX to determine where to break the
% lines. Using \AND forces a line break at that point. So, if LaTeX puts 3 of 4
% authors names on the first line, and the last on the second line, try using
% \AND instead of \And before the third author name.


\author{}
  % examples of more authors
  % \And
  % Coauthor \\
  % Affiliation \\
  % Address \\
  % \texttt{email} \\
  % \AND
  % Coauthor \\
  % Affiliation \\
  % Address \\
  % \texttt{email} \\
  % \And
  % Coauthor \\
  % Affiliation \\
  % Address \\
  % \texttt{email} \\
  % \And
  % Coauthor \\
  % Affiliation \\
  % Address \\
  % \texttt{email} \\


\begin{document}




\begin{abstract}
We study the theoretical and practical limits of $k$-means and $k$-median clustering on large datasets. Since the reader's favorite clustering method is likely
slower than $\tilde{O}(nd)$ time, the general approach is to compress the data and perform the clustering on the compressed representation. Towards this goal,
the number of features can be reduced in effectively linear time with guaranteed accuracy using random projections. Unfortunately, there is no such universal
best choice for compressing the number of points -- while random sampling runs in sublinear time and coresets provide theoretical guarantees, the former does
not enforce accuracy while the latter is too slow as $n$ and $k$ grow. Indeed, it has been conjectured that any sensitivity-based coreset construction
necessitates $\tilde{O}(nd + nk)$ time.

We examine this relationship by first showing that there does exist an algorithm that obtains coresets via sensitivity sampling in $\tilde{O}(nd)$ time --
within log-factors of the time it takes to read the data.  Any approach that significantly improves on this must then resort to practical heuristics, leading us
to consider the spectrum of sampling strategies across both real and artificial datasets in the static and streaming settings. Through this, we show the
conditions in which coresets are necessary for preserving cluster validity as well as the settings in which faster, cruder sampling strategies are sufficient.
As a result, we provide a comprehensive theoretical and practical blueprint for clustering effectively regardless of data size.  Our (anonymized) code is
publicly available at \textcolor{blue}{\href{https://anonymous.4open.science/r/Fast-Coreset-Generation-7564}{source}}.
\end{abstract}

\maketitle


\input{sigmod_introduction}
\input{sigmod_preliminaries}
%
\subsection{Related Work.}

The coreset paradigm has attracted a lot of attention, with a long line of work trying to get the smallest coreset possible in many different metric spaces. The
most prominent example is for Euclidean space \cite{BadoiuHI02, HaM04, Chen09, HuangV20, stoc22}.  In this case, the \textit{group sampling} algorithm developed
in \cite{stoc21, stoc22} yields a coreset of size $\tilde{O}(k\cdot \varepsilon^{-2} \min(k^{z/(z+2)},\varepsilon^{-z}))$ \cite{CLSSS22}, while we know that any
coreset must have size $\Omega \lpar k\eps^{-2}\rpar$ \cite{stoc22}.  Although group sampling has theoretically better bounds than sensitivity sampling, the
experiments of \cite{chrisESA} showed that the later one is likely to be more efficient in practice.

Small size coresets for $k$-median and $k$-means also exist in doubling metrics \cite{huang2018varepsilon}, discrete metrics \cite{FeldmanL11}, metrics induced
by minor-free graphs \cite{BravermanJKW21} or graphs with bounded treewidth \cite{baker2020coresets}.  In finite metrics, the running time $\Omega(nk)$ is
required to compute any approximation to $k$-median or $k$-means \cite{mettu2004optimal}. Since coresets can be used to quickly compute an approximation, the
running time also applies to coreset construction. 

All efficient coreset constructions are probabilistic. This comes with a disadvantage of coresets being difficult to verify and compare. For example, it is
co-NP-hard to check whether a candidate compression is a weak coreset \cite{chrisESA} \footnote{A weak coreset guarantee only requires that a $(1+\varepsilon)$
approximation computed on the coreset yields a $(1+\varepsilon)$ on the entire point set.}. Therefore, although the algorithm succeed with some high
probability, it is unknown how to determine whether the algorithm succeeded or not.  This posed a considerable difficulty for previous experimental evaluations,
where researchers would typically focus on the cost of a solution computed on the designated coreset instead.
%A recent work proposed a pipeline for doing so, and showed that sensitivity sampling algorithm slightly outperforms the theoretically best group sampling
%\cite{chrisESA}.  We will use this pipeline to evaluate the quality of the coresets analyzed in this paper.  We will use a similar heuristic to compo


\section{Fast-Coresets}
\label{sec:theory}

\begin{algorithm}[tb]
   \caption{Fast-Coreset($P, k, \eps, m$)}
   \label{alg:main}
\begin{algorithmic}[1]
   \State {\bfseries Input:} data $P$, number of clusters $k$, precision $\eps$ and target size $m$
   \State Use a Johnson-Lindenstrauss embedding to embed $\tilde P$ of $P$ into $\tilde d = O(\log k)$ dimensions
   \State Find approx. solution $\tilde \calC = \lbra \tilde c_1, ..., \tilde c_k\rbra $ on $\tilde P$ and assignment $\tilde \sigma : \tilde P \rightarrow
   \tilde \calC$ by \fkmeans.	
   \State Let $\calC_i = \tilde \sigma^{-1}(c_i)$. Compute the $1$-median (or $1$-mean) $c_i$ of each $\calC_i$ in $\R^d$.%, and define $\sigma(p) := c_i$ for all $p \in \calC_i$.
   \State For each point $p \in \calC_i$, define
   $s(p) = \frac{\dist^z(p, c_i)}{\cost(\calC_i, c_i)}+ \frac{1}{|\calC_i|}$.
   \State Compute a set $\coreset$ of $m$ points randomly sampled from $P$ proportionate to $s$.
   \State For each $\calC_i$, define $|\hat \calC_i|$ the estimated weight of $\calC_i$ by $\coreset$, namely $|\hat \calC_i| := \sum_{p \in \calC_i \cap
   \coreset} \frac{\sum_{p' \in P}s(p')}{s(p)m}$.
   \State {\bfseries Output:} the coreset $\coreset$, with weights $w(p) = \frac{\sum_{p' \in P}s(p')}{s(p)m} \lpar (1+\eps)|\calC_i| - |\hat \calC_i|\rpar$
\end{algorithmic}
\end{algorithm}


In this section, we first combine two existing results to produce a strong coreset in time $\tilde{O}(nd \log \Delta)$, where $\Delta$ is the spread of
the input.  We show afterwards how to reduce the dependency in $\Delta$ to $\log \log \Delta$, giving the desired nearly-linear runtime.

Our method is based on the following observations about the group sampling \cite{stoc21} and sensitivity sampling \cite{FeldmanL11} coreset construction
algorithms. Both start by computing a solution $\calC$. When $\calC$ is a $c$-approximation, they compute a $c \eps$-coreset of size $\tilde O\lpar
k \eps^{-z-2}\rpar$ and $\tilde O\lpar k \eps^{-2z -2}\rpar$, respectively. Hence, by rescaling, they provide an $\eps$-coreset with size $\tilde O\lpar
k (\eps/c)^{-z-2}\rpar$ and $\tilde O\lpar k (\eps/c)^{-2z-2}\rpar$.  This leads to the following fact:

\begin{fact}\label{fact:logApprox}
Let $\calC$ be an $O\lpar \log^{O(1)} k\rpar$ approximation to $k$-median or $k$-means.
Then, group sampling using solution $\calC$ computes a coreset of size $\tilde O\lpar
k \eps^{-z-2}\rpar$, and sensitivity sampling one of size $\tilde O\lpar k \eps^{-2z-2}\rpar$. 
Both run in time $\tilde O(nd)$, provided $\calC$.
\end{fact}

To turn \cref{fact:logApprox} into an algorithm, we use the quadtree-based \fkmeans approximation algorithm from \cite{cohen2020fast}, which has two key
properties: 
\begin{enumerate}
\item \fkmeans runs in $\tilde O\lpar n d \log \Delta\rpar$ time (Corollary 4.3 in \cite{cohen2020fast}), and
\item \fkmeans computes an assignment from input points to centers that is an $O\lpar d^z \log k\rpar$ approximation to $k$-median ($z=1$) and $k$-means ($z=2$)
    (see Lemma 3.1 in \cite{cohen2020fast} for $z=2$ and the discussion above that lemma for $z=1$). Applying dimension reduction techniques \cite{MakarychevMR19}, the dimension
    $d$ may be replaced by a $\log k$ in time $\tilde O(nd)$. This results in a $O\lpar\log^{z+1} k\rpar$ approximation. 
\end{enumerate}

The second property is crucial for us: the algorithm does not only compute centers, but also assignments in $\tilde{O}(nd\log \Delta)$ time.  We describe how to
combine it with sensitivity sampling in \cref{alg:main} and prove in \cref{app:theory} that this computes an $\eps$-coreset in time $\tilde O(nd \log \Delta)$.

\begin{corollary}\label{cor:mainAlg}
\cref{alg:main} runs in time $\tilde O\lpar n d \log \Delta\rpar$ and computes an $\eps$-coreset for $k$-means.
\end{corollary}
Furthermore, we generalize \cref{alg:main} to other fast $k$-median approaches in \cref{app:extensions}.
Thus, it is easy to combine existing results to obtain an $\eps$-coreset without an $\tilde{O}(nk)$ time-dependency.  However, our method thus far has only
replaced the $\tilde{O}(nd + nk)$ runtime by $\tilde{O}(nd \log \Delta)$. Indeed, the spirit of the issue remains -- this is not near-linear in the input size.
We verify this by devising a dataset that has $n - n'$ points uniformly in the $[-1, 1]^2$ square. Then, for $r \in \mathbb{Z}^+$, we produce a sequence of
points at $(0, 1), (0, 0.5), \cdots, (0, 0.5^r)$ and copy this sequence $n' / r$ times, each time with a different $x$ coordinate. The result is a dataset of
size $n$ where $\log \Delta$ grow linearly with $r$ (and, therefore, linearly with $n$). The resulting linear time-dependency can then be seen in Table \ref{tbl:logdelta}.

\section{Reducing the Impact of the Spread}
\label{sec:logdelta}
\newcommand{\boxsize}{\textsc{MaxDist}}

\paragraph*{Overview of the Approach}
We assume in this section that the smallest pairwise distance is at least $1$, and $\Delta$ is a (known) upper-bound on the diameter of the input.
To remove the $\log\Delta$ dependency, we proceed in two steps: first, we compute a very crude upper-bound on the cost $U$ of the optimal solution -- up to
a $\poly(n)$ factor.  If $U$ is a $c$-approximation of the optimal cost then the minimum distance can be reduced by rounding all coordinates to multiples of $g
= U/(cn)$, giving us a minimum distance of no less than $g$; it is then enough to reduce the diameter to $\poly(n) U$.  For this, we place a grid with cell length
$O(n \cdot U)$ centered at a random location, so that two points from the same cluster in $\opt$ fall into the same cell w.h.p. This implies that distinct cells do not interact with each other
in any reasonable solution.  Then, we diameter of the input by "moving" non-empty cells closer to each other. We will focus this section on the simpler $k$-median
problem but show how to reduce $k$-means to this case in \cref{app:redKM}. We also provide rigorous proofs in \ref{}, and give here an overview.


\paragraph*{Computing a crude upper-bound}

As described, we start by computing an approximate solution $U$ such that $\opt \leq U \leq \poly(n) \cdot \opt$. For this, the first step is to embed the input
into a quadtree. This embedding has two key properties: first, distances are preserved up to a multiplicative factor $O(\sqrt{d} \log \Delta)$, and therefore the $k$-median cost is preserved up to this factor as well. Second, the metric is a \emph{hierarchically separated tree}: it can be represented with a tree, where points of $P$ are the leafs, and the distance between two points is given by the depth of their lowest common ancestor: if it is at depth $\ell$, their distance is $\Delta 2^{-\ell}$. 
 Our first lemma shows
that finding the first level of the tree for which the input lies in $k+1$ disjoint subtrees provides us with the desired approximation. 

\begin{lemma}\label{lem:apxTree}
[Proof in \cref{app:apx-tree-proof}]Let $\ell$ be the first level of the tree such that at least $k+1$ subtrees at level $i$ contain any point. Then, $\sqrt{d}2^{-ell+1} \cdot \Delta \leq
\opt_T \leq n \cdot \sqrt{d}2^{-ell+4} \cdot \Delta$.
\end{lemma}

We prove this in \cref{app:apx-tree-proof} of the appendix. A direct consequence  is that the first level of the tree for which at least
$k+1$ cells are non empty allows to compute easily an $O(n)$-approximation to $k$-median on the tree metric. Since the tree metric approximates the oringial Euclidean metric up to $O(\sqrt{d} \log \Delta)$, this is therefore an $O(n \sqrt{d} \log \Delta)$-approximation to $k$-median in the Euclidean space.
 To turn that observation into an algorithm, one needs to count the number of non-empty cells at a given level $\ell$: for each point, identifying the cell that contains it can be done using modulo operations, and counting the number of distinct non-empty cells can be done with a mere dictionnary.
This is done in time $\tilde O(nd)$, and a pseudo code is given \cref{alg:crudeApx}, in \cref{app:pseudoCode}.  Using a binary search on the $O(\log \Delta)$ many levels then concludes this section with the following result:

\begin{lemma}\label{lem:crudeApx}[Proof in \cref{app:apx-tree-proof}]
There is an algorithm running in time $\tilde O(nd \log \log \Delta)$ that computes an $O(n \sqrt{d} \log \Delta)$-approximation to $k$-median, and $O(n^2 d \log \Delta)$-approximation to $k$-means.
\end{lemma}




\paragraph*{From Approximate Solution to Reduced Spread}


Let $U$ be an upper-bound on the optimal cost, computed via \cref{lem:crudeApx}. We place a grid with side length $r:= d n^2\cdot U$, centered at a random point in $\{0, ..., r\}^d$.
The following folklore lemma ensures that with high probability (over the randomness of the center of the grid), no cluster of the optimal solution is spread on several grid cells.

\begin{lemma}[Find a reference]
The probability that two points $p$ and $q$ are in different grid cells is $O\lpar \frac{\|p-q\|^2}{n^2 U}\rpar$
\end{lemma}

Since $U$ is larger than the distance between any input point and its center in the optimal solution, a union-bound ensures that with probability $1-1/n$, no
cluster of this solution is split among different cells.  In particular, there are at most $k$-non empty cells. We call those "boxes".

From this input, we build a new set of points $P'$ as follows: first, identify the non empty cell (using a dictionnary as previously). We associate each box with
its middle point.  For each coordinate $i \in \lbra 1, ..., d \rbra$, sort the $k$ centers according to their value on coordinate $i$. Then, for
each $j \in \lbra 1, ..., k\rbra$, let $c^i_j$  be the $i$-th coordinate of centers of the $j$-th box. If $c^i_{j+1} - c^i_j
\geq 2r$, then for all boxes $j'$ with $j' > j$, shift the points of $j'$ by $c^i_{j+1} - c^i_j - 2r$ in the $i$-th coordinate. This can be implemented in near-linear time, as described in \cref{alg:reduce-diam} (presented in \cref{app:pseudoCode}). The dataset $P'$ obtained after these transformations have the following properties:
%This can be implemented with a linear scan and has the two following effects:
%: first, the diameter of the input is now $\sqrt{d} \cdot 3r \cdot k$, as they are at most $k$ boxes of length $r$, each separated by at most $2r$.  Second, two boxes that were adjacent before the transformation are still adjacent after and two boxes that were non-adjacent are still non-adjacent.


\begin{proposition}\label{prop:boxes}
In $P'$, the diameter of the input is $\sqrt{d} \cdot 3d n^2\cdot U \cdot k$. 
Furthermore, two boxes that are adjacent (respectively non-adjacent) in $P$  are still adjacent (resp. non-adjacent) in $P'$.
\end{proposition}
\begin{proof}
In $P'$, along any
coordinate the maximal distance between the centers of two boxes is $2r = 2d n^2\cdot U$. Since there are at most $k$ boxes, the total distance along a coordinate is at most $2kr$, and therefore the diameter of the whole point set is $\sqrt{d} \cdot 2kr$.

If two boxes are adjacent, then they will be adjacent along any coordinate and their points have same shift in all coordinates. Therefore, the corresponding boxes are adjacents in $P'$. If they are not adjacent, there is at least one coordinate where they are at distance at least $2r$: in this case, their shift will be different, and they will stay non-adjacent.
\end{proof}


The first property allows us to reduce the spread to $(nd \log \Delta)^{O(1)}$.  Indeed, one can round all coordinates of points in $P'$ to the closest multiple of
$g := \frac{U}{n^4 d^{2} \log \Delta}$.
Combined with the diameter reduction, this ensures that the spread of the dataset obtained is at most $(nd \log \Delta)^{O(1)}$. 
Furthermore, the second property of \cref{prop:boxes} combined with the choice of $g$ ensures that the cost of any reasonable solution is the same before and after the transformation, as stated in the following lemma:

\begin{lemma}\label{lem:reduceSpread}
Let $P'$ be the outcome of the diameter reduction and rounding steps. $P'$ has spread $(nd \log \Delta)^{O(1)}$.

Suppose $U$ is such that $\opt \leq U \leq c \opt$, and 
let $\calS'$ be a $c'$-approximation for  $k$-median (resp. $k$-means) on $P'$. 
Then, one can compute a solution for $P$ for $k$-median (resp. $k$-means) on $P$, with same cost as $\calS'$ for $P'$ up to an additive error $\opt / n$, in time $O(nd)$. The same holds by reversing the roles of $P$ and $P'$.
\end{lemma}
\begin{proof}
First, in rounding points to the closest multiple of $g$,
  the distance between any point and its rounding is at most $g \leq \frac{\opt}{n^2}$. Summing over all points,
any solution computed on the gridded data has cost within an additive factor $\pm \frac{\opt}{n}$ of the true cost. 

Now, since distances in $P'$ are smaller than in $P$ (up to an additive $\frac{\opt}{n^2}$ due to the rounding), the optimal solution for $P'$ has cost at most $U$. Therefore, two points that are in non-adjacent boxes
(i.e., at distance more than $d n^2\cdot U$) are not in the same cluster of $\calS'$ as otherwise $\calS'$ would not be a $c'$-approximation. 
 Let $\calS$ be
the solution obtained from $\calS'$ by reversing the construction of $P'$, namely re-adding the shift that was substracted to every box. Since adjacency is preserved, all clusters end up having the same shift, and therefore all
intra-cluster distances are the same in $P$ and $P'$. Therefore, the costs are equal.

Finally, the smallest
non-zero distance is $g = \frac{U}{n^4 d^{2} \log \Delta}$, and the diameter is $\sqrt{d} \cdot 3d n^2\cdot U \cdot k$ (see \cref{lem:boxes}), implying that the spread of $P'$ is $(nd \log \Delta)^{O(1)}$.
\end{proof}



%The second property, on the other hand, ensures that we can transform a solution $\calS'$ for $P'$ to a solution with exactly the same cost for $P$: in any
%(reasonable) solution, points from two non-adjacent boxes will not be in the same cluster in either $P'$ or $P$. Therefore, simply adding back the shift to
%centers of $\calS'$ allows us to transform it to a solution $\calS$. We note that the distance between any point and its closest center does not change. This is
%formalized in the next lemma.

Combining the algorithm of \cref{lem:crudeApx}, which gives a bound on $U$, with \cref{lem:reduceSpread} concludes the section with the following theorem:

\begin{theorem}
Given $P \subset \R^d$ with spread $\Delta$, there is an algorithm running in time $O(nd \log \log \Delta)$ that computes a set $P'$ such that (1) the spread of $P'$ is $\poly(n,d, \log(\Delta))$ and (2) any solution that is a $c'$-approximation for  $k$-median (resp. $k$-means) on $P'$ can be converted in time $O(nd)$ into a solution with same cost on $P$, up to an additive error $O(\opt / n)$.
\end{theorem}

%\section{Reducing the Impact of the Spread}
To reduce the dependency of $\Delta$ on the running time, we proceed in two steps: first, we compute a very crude upper-bound on the cost $U$ of the optimal solution -- up to a $\poly(n)$ factor. 
If $U$ is a $c$-approximation of the optimal cost, the natural attempt to reduce the aspect-ratio is to round all coordinates to multiples of $U/(cnà$, such that the smallest distance is $U/(cn)$; it is then enough to reduce the diameter to $\poly(n) U$. We proceed as follows: we place a grid with cell length $O(n \cdot U)$, so that two points from the same cluster in $\opt$ fall into the same cell w.h.p. That way, the distinct cells do not interact with each other in any reasonable solution. 
Then, we compress the input by "moving" non-empty cells closer to each other.

\subsection{Computing a crude upper-bound}
As described, we start by computing a $U$ such that $U \leq \poly(n) \cdot \cost(\opt)$. For this, we start by simplifying a lot our task: our first lemma shows that it is enough focus on the $k$-median problem; we then embed the input into a tree, easier to handle than Euclidean space. In this tree, we will show that it is enough to find the first level for which the input lies in $k+1$ disjoint subtrees: the final algorithm is therefore a mere binary-search to find efficiently this level.

\paragraph{Reduction to $k$-median.}

\begin{lemma}
Let $\calS$ be a $c$-approximation for $k$-median on $P$. Then, $\calS$ is a $nc^2$-approximation for $k$-means on $P$.
\end{lemma}
\begin{proof}
Let $\cost_1$ (resp. $\cost_2$) be the $k$-median (resp. $k$-means) cost, $\opt_1$ (resp. $\opt_2$) be the optimal $k$-median (resp. $k$-means) solution. We have the following inequalities:
\begin{align*}
\cost_2(\calS) &= \sum_{p\in P} \dist(p, \calS)^2 \leq \lpar \sum_{p\in P} \dist(p, \calS)\rpar^2\\
&\leq c^2 \cdot \lpar  \sum_{p\in P} \dist(p, \opt_1)\rpar^2\\
&\leq c^2 \cdot \lpar  \sum_{p\in P} \dist(p, \opt_2)\rpar^2\\
&\leq c^2 \cdot n \cdot  \sum_{p\in P} \dist(p, \opt_2)^2,
\end{align*}
where the last inequality stems from Cauchy-Schwarz. Therefore, $\calS$ is a $nc^2$ - approximation to $k$-means. 
\end{proof}

\paragraph{Reduction to Structured Tree.}
\newcommand{\boxsize}{\textsc{MaxDist}}
It is well known that, for $k$-median, embedding the input into a tree using the quadtree decomposition loses only a $O(d \log \Delta)$ factor. Building explicitely this embedding takes time $O(nd \log \Delta)$: however, we will show that we only need to build only a fraction of it, allowing for a running time $O(nd \log \log \Delta)$. 

The embedding is constructed as follows. First, the algorithm finds a box enclosing all input points, centered at zero, with all side length equal to $\boxSize$.\footnote{This can be done as follows: select an arbitrary input point, and translate the dataset so that this point is at the origin. Then, compute the maximum distance from any point to the origin in time $O(nd)$, and let $\boxSize$ be that distance.} Then, a random shift $s \leq \boxsize$ is added to all coordinates of all points: the input is now in the box $[-2\boxsize, 2\boxsize]^d$ and this transformation did not change any distances, henceforth the $k$-median cost is preserved. 
The $i$-th level of the tree (for $i \in \lbra 0, ..., \log \Delta \rbra$) is constructed as follows: a grid of side length $2^{-i} \cdot 2\boxsize$ centered at $0$ is placed, and each cell is a node in the tree.
 The parent of a cell $c$ is simply the cell at level $i-1$ that contains $c$, and the distance between $c$ and its parent is set to $\sqrt{d}2^{-i} \cdot 2\boxsize$ (namely, the $\ell_2$ diameter of $c$).
 
 This tree has $O(\log \Delta)$ levels, and it is known that for any solution $\calS$, its cost in the tree metric is within a factor $O(\sqrt d \log \Delta)$ of its cost for the original metric. Therefore, it is enough for us to find an approximation in the tree metric. For this, we let $\opt_T$ be the optimal solution in the tree metric, and have the following lemma:
 
\begin{lemma}\label{lem:apxTree}
Let $i$ be the first level of the decomposition such that at least $k+1$ cells at level $i$ contains any point. Then, $\sqrt{d}2^{-i+1} \cdot \boxsize \leq \opt_T \leq n \cdot \sqrt{d}2^{-i+4} \cdot \boxsize$.
\end{lemma}
\begin{proof}
If the input is spread in $k+1$ distinct cell at level $i$, then in any solution there is at least one of those cells with no center. In the tree metric, points lying in this cell have therefore connexion cost at least $\sqrt{d}2^{-i+1} \cdot \boxsize$, and thus the left-hand-side of the inequality holds.

On the other hand, if the input is contained into $k$ cells at level $i-1$, then placing arbitrarily a center in each cell yields a solution with cost at most $n \cdot \sqrt{d}2^{-i+4} \cdot \boxsize$: indeed, the distance from any point to its closest center is at most $2 \cdot \sum_{j \geq i-1} \sqrt{d}2^{-j} \cdot 2\boxsize \leq 4 \sqrt{d}2^{-i+2} \cdot \boxsize$ (summing the edge length from the point to the cell at level $i$, and then going down to the center). This concludes.
\end{proof}

A consequence of \cref{lem:apxTree} is that the first level of the tree for which at least $k+1$ cells are non empty provides an $O(n)$-approximation. 
To count the number of non-empty cells at a given level $i$, one can merely iterate over all points, for each point identify the cell that contains it (using modulo operation), and store all those cells into a hash table to count the number of elements. This is done in time $O(nd)$.
Using a binary search on the $O(\log \Delta)$ many levels then concludes this section with the following result:

\begin{lemma}\label{lem:crudeApx}
There is an algorithm running in time $O(nd \log \log \Delta)$ that computes an  $O(n^2 d \log \Delta)$-approximation to $k$-median or $k$-means.
\end{lemma}

\subsection{From Approximate Solution to Small Aspect-Ratio}
Let $U$ be an upper-bound on the optimal cost that is a $c$-approximation, computed via \cref{lem:crudeApx}. We place a grid with side length $d n^2\cdot U$.
The following folklore lemma ensures that with high probability, no cluster of the optimal solution is spread on several grid cells.

\begin{lemma}
The probability that two points $p$ and $q$ are in different grid cells is $O(\frac{\|p-q\|^2}{n^2 U}$
\end{lemma}
Since $U$ is larger than the distance between any input point and its center in the optimal solution, a union-bound ensures that with probability $1-1/n$, no cluster of this solution is split among different cells.
In particular, there are at most $k$-non empty cells. We call those "boxes".

From this input, we build a new one $P'$ as follows: first, identify the non empty cell (using a hash table as previously). We associate each box with its center.
For each coordinate $i \in \lbra 1, ..., d \rbra$, sort (in time $O(k \log k)$ the centers according to their value on coordinate $i$. Then, for each $j \in \lbra 1, ..., k\rbra$, let $c^i_j$ and $c^i_{j+1}$ be the $i$-th coordinate of centers of the $j$-th and $(j+1)$-th boxes. If $c^i_{j+1} - c^i_j \geq 2d n^2\cdot U$, then for all cell $j'$ with $j' > j$, shift points  of $j'^$ by an amount $c^i_{j+1} - c^i_j - 2d n^2\cdot U$ in $i$-th coordinate.


This can be implemented with a linear scan, and has two effects: first, the diameter of the input is now $\sqrt{d} \cdot 2d n^2\cdot U \cdot k$, as along any coordinate the maximal distance is $2d n^2\cdot U \cdot k$. Second, two boxes that were adjacents are still adjacents, and two boxes that were non-adjacents are still non-adjacent.

The first property allows to reduce the spread to $(nd \log \Delta)^{O(1)}$. Indeed, one can round all coordinates to the closest multiple of $\frac{U}{n^4 d^{2} \log \Delta}$. That way, each point is at distance at most $\frac{U}{n^4 d^{2} \log \Delta}$ of its rounding. using \cref{lem:crudeApx}, $U \leq n^2 d \log(\Delta) \opt$, and therefore the distance between any point and its rounding is at most $\frac{\opt}{n^2}$. Summing this error over all points, this ensures that any solution computed on the rounding has cost within an additive factor $\pm \frac{\opt}{n}$ of the true cost. Furthermore, the smallest non-zero distance is $\frac{U}{n^4 d^{2} \log \Delta}$, and therefore the spread of the new metric is $2 n^6 d k \log \Delta = (nd \log \Delta)^{O(1)}$, as claimed.

The second property, on the other hand, ensures that we can transform a solution $\calS'$ for $P'$ to a solution with exactly the same cost for $P$: in any (reasonable) solution, points from two non-adjacent box will not be in the same cluster in $P'$ nor in $P$. Therefore, simply adding to centers of $\calS'$ the corresponding shift allows to transform it to a solution $\calS$, such that the distance between any point and its closest center does not change. This is formalized in the next lemma.

\begin{lemma}
Let $\calS'$ be a $c'$-approximation for  $k$-median (resp. $k$-means) on $P'$, where $c' \leq nc$ and $c$ is the approximation guarantee from \cref{lem:crudeApx}. Then, one can compute a solution for $P$ for $k$-median (resp. $k$-means) on $P$, with same cost as $\calS'$ for $P'$, in time $O(nd)$.
\end{lemma}
\begin{proof}
First, since distances in $P'$ are smaller than in $P$, the optimal solution for $P'$ has cost at most $U$. Therefore, two points that are in non-adjacent boxes (i.e., at distance more than $d n^2\cdot U$) are not in the same cluster of $\calS'$ -- as otherwise $\calS'$ would not be a $c'$-approximation.

Let $\calS$ be the solution obtained from $\calS'$ by reversing the construction of $P'$. Since this construction preserves adjacency, for all cluster of the solution, all distances are the same in $P$ and $P'$: therefore, the cost are alike. This concludes.
\end{proof}


\input{sigmod_results}
\input{sigmod_conclusion}



% Acknowledgements should only appear in the accepted version.

\bibliographystyle{plainnat}
\bibliography{references}

\appendix
\input{sigmod_appendix}
\end{document}
