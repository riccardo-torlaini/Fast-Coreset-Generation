\section{Preliminaries and Related Work}

\subsection{On Sampling Strategies.}
\label{ssec:sens_sampling}

As discussed, we focus our study on linear- and sublinear-time sampling strategies. Specifically, given a dataset $P \in \R^{n \times d}$, we want to sample $S
\in \R^{m \times d} \subset P$ such that $m \ll n$ along with a weights vector $w \in \R^m$. The goal is then that for some solution $\calC$, $S$
provides us with an idea of the solution's quality with respect to the original dataset, i.e. $\sum_{p \in S} w_p \cost(p, \calC) \sim \sum_{p \in P} \cost(p,
\calC)$. 
The quickest sampling strategy, running in sublinear time, is uniform sampling. It is clear, however, that this cannot provide any cost-preservation guarantee
as missing a single extreme outlier will cause the sampling strategy to fail. Thus, any approach that outperforms this strategy must read in the entire dataset
and therefore run in at least linear time. Indeed, sublinear algorithms always require some assumption on the input to provide guarantees, see \cite{Ben-David07,czumaj2007sublinear,HJJ23,meyerson2004k}.

Among these more sophisticated sampling strategies, one cannot do better than the strong coreset guarantee. Specifically, a strong $\eps$-coreset is a subset $S
\subseteq P$ with weights $\tilde w$ such that for \emph{any} solution $\calC$, \[\sum_{p \in S} \tilde w_p \cost(p, \calC) \in (1 \pm \eps) \cost(P, \calC)\] with
high probability.  Going forward, we will discuss this in the context of the $k$-median and $k$-means cost functions: for a dataset $P \in \R^d$ with weights $w
: P \rightarrow \R^+$, and any $k$-tuple $\calC$ in $\R^d$, \[\cost_z(P, \calC) := \sum_{p \in P} \tilde w(p) \dist^z(p, \calC),\] with
$z=1$ for $k$-median and $z=2$ for $k$-means. We use $\opt$ to denote $\min_{\calS} \cost_z(P, \calC)$.

Recently, sampling with respect to sensitivity values has grown to prominence due to its simplicity and coreset guarantee.  True sensitivity values are defined
as $\sup_{\calC} \frac{\dist^z(p, \calC)}{\cost_z(P, \calC)}$, where the supremum is taken over all possible solutions $\calC$. Intuitively, this is a measure
of the maximum impact a point can have on a solution and is difficult to evaluate directly.
Thus, the approximate sensitivity-sampling algorithm we consider is the following, as introduced in \cite{FeldmanL11}.
Given a solution to a clustering problem $\calC$, importance scores are defined as
\begin{equation}
\label{eq:sensitivity}
\sigma_\calC(p) = \dfrac{\cost(p, \calC)}{\cost(\calC_{p}, \calC)} + \dfrac{1}{|\calC_p|},
\end{equation}
where $\calC_p$ is the cluster that $p$ belongs to. This is always an upper-bound on the sensitivity values, see \cite{FL11,FeldmanSS20}.

The coreset $S$ then consists of $m$ points sampled proportionate to $\sigma$ with weights defined as follows. First, for any sampled point $p$, define $w_p :=
\frac{1}{\Pr[p \in S]} = \frac{\sum_{p'} s_\calC(p')}{m s_\calC(p)}$. For a cluster $C_i$, let $|\hat{C_i}|$ be the number of points in $C_i$ estimated by the
sample, i.e. $C_i \cap S$ weighted with $w_p$. A sampled point $p$ in $C_i$ is
weighted $\tilde w(p) = \tilde w_1(p) \lpar (1+\eps)|C_i| - |\hat{C_i}|\rpar$.  \cite{FeldmanL11} and subsequent works showed that, when $\calC$ is
a $O(1)$-approximation, sampling $S = \tilde O\lpar k \eps^{-2z-2}\rpar$ many points was enough to ensure concentration, namely, $S$ is a coreset with
probability at least $2/3$.

To perform this algorithm, the bottleneck in the running time lies in computing the solution $\calC$ as well as then obtaining costs of every point to its
assigned center in $\calC$. Using any bicriteria approximation algorithm\footnote{For $k$-means an $(\alpha,\beta)$ bicriteria approximation is an algorithm
that computes a solution $\calC$ satisfying $\cost(P\calC)\leq \alpha\cdot \opt$ and $|\calC|\leq \beta\cdot k$.} such as the standard $k$-means++ algorithm
\cite{ArV07} combined with dimension reduction techniques (see for example \cite{BecchettiBC0S19,CohenEMMP15,MakarychevMR19}), this takes time $\tilde O(nk
+nd)$. This is precisely what was conjectured as the necessary runtime for obtaining $k$-means and $k$-median coresets, as merely assigning points to their
centers from the bicriteria seems to require $\Omega(nk)$ running time.

\begin{figure}
\centering
\begin{tabular}{c}
    \hskip-0.22cm \includegraphics[width=.98\linewidth]{images/1/coreset_distortion-m_scalar_for_sens_sampling.pdf} \\
    \includegraphics[width=.95\linewidth]{images/2/coreset_runtime-Effect_of_k_for_sens_sampling.pdf}
\end{tabular}
\caption{\emph{Top}: the effect of the $m$-scalar parameter on the distortion metric for sensitivity sampling approaches.
\emph{Bottom}: the effect of varying $k$ for the sensitivity sampling approaches.}
\label{fig:coreset_size_on_sens_quality}
\end{figure}


\subsection{Other Coreset Strategies}
\label{ssec:clustering_prelim}

Much of the advancements regarding coresets have sought the smallest coreset possible across metric spaces and downstream objectives. The most prominent
examples are in Euclidean space \cite{BadoiuHI02, HaM04, Chen09, HuangV20, stoc22}, with much of the focus on obtaining the optimal size in the $k$-means and
$k$-median setting. Recently, a lower bound \cite{huangLB} showed that the group sampling algorithm developed in \cite{stoc21, stoc22} is optimal.

Although this algorithm yields a coreset of size $\tilde{O}(k\cdot \varepsilon^{-2}
\min(k^{z/(z+2)},\varepsilon^{-z}))$ \cite{CLSSS22} and provides theoretically smaller coresets than sensitivity sampling, the experiments of \cite{chrisESA}
showed that the latter is often more efficient in practice. We also note that one could use virtually any coreset construction as many are linear-time once
provided with an initial solution $\calC$ and assignment $\sigma$.  Therefore, our algorithm could use as a subroutine any coreset construction, if in the
future one is developed that is preferable to sensitivity and group sampling.

In terms of other linear-time methods, we are only aware of the lightweight coresets approach\cite{BachemL018}, wherein one performs sensitivity
sampling with respect to a solution $\calC=\{\mu\}$, i.e. the mean of the data set. This runs in $O(nd)$ time but provides a weaker guarantee -- one incurs an
additive error of $\varepsilon\cdot \cost(P,\{\mu\})$.  We note that this can be generalized to performing sensitivity sampling using a $\calC$ that has fewer
than $k$ centers. Since the lightweight coreset construction uses the $1$-means solution and sensitivity sampling uses an $O(1)$ approximation to the
$k$-means/median solution, it is natural to investigate the relationship between the approximation quality and coreset quality. We discuss this in more depth in
Section \ref{ssec:algorithms}.

All efficient coreset constructions are probabilistic. This comes with a disadvantage of coresets being difficult to evaluate. For example, it is
co-NP-hard to check whether a candidate compression is a weak coreset \cite{chrisESA} \footnote{A weak coreset guarantee only requires that a $(1+\varepsilon)$
approximation computed on the coreset yields a $(1+\varepsilon)$ on the entire point set.}. Therefore, although coreset algorithms succeed with some high
probability, it is unclear how to computationally verify this.  This posed a considerable difficulty for previous experimental evaluations,
where researchers would typically focus on the cost of a solution computed on the designated coreset instead. We refer to \cite{chrisESA} for further discussion
on this topic and discuss our metrics in Section \ref{sssec:metrics}.


\subsection{Quadtree Embeddings}

We give a brief description of the quadtree metric embedding procedure here and refer to \cref{app:quadtree} for a more formal explanation.  The central idea is
that each hypercube in the input space can be split into $2^d$ sub-cubes and that this can be represented in a tree structure. Setting the weight of each branch
equal to the length of the parent node's diagonal then guarantees that the expected distance in the tree is within a factor $O(\log n \sqrt{d} \log
\Delta)$\andrew{There's a $\log n$ on the expected distortion, right? This should also hold for the $k$-median setting?} of the distance in the original space,
where $\log \Delta$ is the maximum depth of the tree. Lastly, this tree can be constructed in $\tilde{O}(nd \log \Delta)$ time.
