\section{Preliminaries and Related Work}
\label{sec:preliminaries}

\subsection{On Sampling Strategies.}
\label{ssec:sens_sampling}

As discussed, we focus our study on linear- and sublinear-time sampling strategies. Generally speaking, we consider compression algorithms through the lens of
three requirements: 
\begin{itemize}
    \item One can find a compression in the time it takes to read in the data.
    \item The size of the compressed dataset does not depend on the size of the original data.
    \item Solutions on the compressed dataset are provably good on the original data.
\end{itemize}
If these requirements are satisfied then, when analyzing runtimes on large datasets, it is always preferable to compress the dataset and then perform the task
in question on the compressed representation.

Specifically, given a dataset $P \in \R^{n \times d}$, we concern ourselves with sampling $\Omega \in \R^{m \times d} \subset P$ (such that $m \ll n$) along
with a weights vector $w \in \R^m$. The goal is then that for some solution $\calC$, $\Omega$ provides us with an idea of the solution's quality with respect to
the original dataset, i.e. $\sum_{p \in \Omega} w_p \cost(p, \calC) \sim \sum_{p \in P} \cost(p, \calC)$.

The quickest sampling strategy, running in sublinear time, is uniform sampling. It is clear, however, that this cannot provide any cost-preservation guarantee
as missing a single extreme outlier will cause the sampling strategy to fail. Thus, any approach that outperforms this strategy must read in the entire dataset
and therefore run in at least linear time. Indeed, sublinear algorithms always require some assumption on the input to provide guarantees, see
\cite{Ben-David07,czumaj2007sublinear,HJJ23,meyerson2004k}.

Among these more sophisticated sampling strategies, coresets offer the strongest compression guarantee. Specifically, a strong $\eps$-coreset is an $\Omega
\subseteq P$ with weights $\tilde w$ such that for \emph{any} solution $\calC$, \[\sum_{p \in \Omega} \tilde w_p \cost(p, \calC) \in (1 \pm \eps) \cost(P,
\calC)\] with high probability.  Going forward, we will discuss this in the context of the $k$-median and $k$-means cost functions: for dataset $P \in \R^d$
with weights $w : P \rightarrow \R^+$, and any $k$-tuple $\calC$ in $\R^d$, \[\cost_z(P, \calC) := \sum_{p \in P} \tilde w(p) \dist^z(p, \calC),\] with $z=1$
for $k$-median and $z=2$ for $k$-means. We use $\opt$ to denote $\min_{\calC} \cost_z(P, \calC)$ and will denote an $\alpha$-approximation as any solution
$\calC$ such that $\cost(\calC) \leq \alpha \cdot \opt$.

Recently, sampling with respect to sensitivity values has grown to prominence due to its simple-to-obtain coreset guarantee.  True sensitivity values are defined
as $\sup_{\calC} \frac{\dist^z(p, \calC)}{\cost_z(P, \calC)}$, where the supremum is taken over all possible solutions $\calC$. Intuitively, this is a measure
of the maximum impact a point can have on a solution and is difficult to evaluate directly.
Thus, the approximate sensitivity-sampling algorithm we consider is the following, as introduced in \cite{FL11}.
Given an $\alpha$-approximate solution to a clustering problem $\calC$, importance scores are defined as
\begin{equation}
\label{eq:sensitivity}
\sigma_\calC(p) = \alpha \cdot \left( \dfrac{\cost(p, \calC)}{\cost(\calC_{p}, \calC)} + \dfrac{1}{|\calC_p|} \right),
\end{equation}
where $\calC_p$ is the cluster that $p$ belongs to. This is always an upper-bound on the sensitivity values, see \cite{FL11}.

The coreset $\Omega$ then consists of $m$ points sampled proportionate to $\sigma$ with weights defined as follows. First, for any sampled point $p$, define $w_p :=
\frac{1}{\Pr[p \in \Omega]} = \frac{\sum_{p'} s_\calC(p')}{m s_\calC(p)}$. For a cluster $C_i$, let $|\hat{C_i}|$ be the number of points in $C_i$ estimated by the
sample, i.e. $C_i \cap \Omega$ weighted with $w_p$. A sampled point $p$ in $C_i$ is
weighted $\tilde w(p) = \tilde w_1(p) \lpar (1+\eps)|C_i| - |\hat{C_i}|\rpar$.  \cite{HuangV20} and subsequent works showed that, when $\calC$ is
a $O(1)$-approximation, sampling $\Omega = \tilde O\lpar k \eps^{-2z-2}\rpar$ many points was enough to ensure concentration, namely, $\Omega$ is a coreset with
probability at least $2/3$.

To perform this algorithm, the bottleneck in the running time lies in computing the solution $\calC$ as well as then obtaining costs of every point to its
assigned center in $\calC$. Using any bicriteria approximation algorithm\footnote{For $k$-means an $(\alpha,\beta)$ bicriteria approximation is an algorithm
that computes a solution $\calC$ satisfying $\cost(P\calC)\leq \alpha\cdot \opt$ and $|\calC|\leq \beta\cdot k$.} such as the standard $k$-means++ algorithm
\cite{ArV07} combined with dimension reduction techniques (see for example \cite{BecchettiBC0S19,CohenEMMP15,MakarychevMR19}), this takes time $\tilde O(nk
+nd)$. This is precisely what was conjectured as the necessary runtime for obtaining $k$-means and $k$-median coresets, as merely assigning points to their
centers from the bicriteria seems to require $\Omega(nk)$ running time, see \cite{DSWY22}. 

\begin{figure}
\centering
\begin{tabular}{c}
    \hskip-0.22cm \includegraphics[width=.98\linewidth]{images/1/coreset_distortion-m_scalar_for_sens_sampling.pdf} \\
    \includegraphics[width=.95\linewidth]{images/2/coreset_runtime-Effect_of_k_for_sens_sampling.pdf}
\end{tabular}
\caption{\emph{Top}: the effect of the $m$-scalar parameter on the distortion metric for sensitivity sampling approaches.
\emph{Bottom}: the effect of varying $k$ for the sensitivity sampling approaches.}
\label{fig:coreset_size_on_sens_quality}
\end{figure}


\subsection{Other Coreset Strategies}
\label{ssec:clustering_prelim}

Much of the advancements regarding coresets have sought the smallest coreset possible across metric spaces and downstream objectives. The most prominent
examples are in Euclidean space \cite{BadoiuHI02, HaM04, Chen09, HuangV20, stoc22}, with much of the focus on obtaining the optimal size in the $k$-means and
$k$-median setting. Recently, a lower bound \cite{huangLB} showed that the group sampling algorithm developed in \cite{stoc21, stoc22} is optimal.

Although these coresets have size $\tilde{O}(k\cdot \varepsilon^{-2} \min(k^{z/(z+2)},\varepsilon^{-z}))$ \cite{CLSSS22} and are theoretically smaller than
those obtained by sensitivity sampling, the experiments of \cite{chrisESA} showed that the latter is often more efficient in practice. We also note that one
could use virtually any coreset construction as many are linear-time once provided with an initial solution $\calC$ and assignment $\sigma$.

In terms of other linear-time methods with sensitivity sampling, we are only aware of the lightweight coresets approach (\cite{BachemL018}), wherein one
samples with respect to a solution $\calC=\{\mu\}$, i.e. the mean of the data set. This runs in $O(nd)$ time but provides a weaker guarantee -- one incurs an
additive error of $\varepsilon\cdot \cost(P,\{\mu\})$.  We note that this can be generalized to performing sensitivity sampling using a $\calC$ that has fewer
than $k$ centers. We discuss this in more depth in Section \ref{ssec:algorithms}.

Lastly, the BICO coreset algorithm \cite{bico} utilizes the SIGMOD test-of-time winning BIRCH \cite{birch} clustering method to obtain $k$-means coresets. While
BICO was developed to perform quickly in a data-stream, we show in Section \ref{sec:results} that it does not provide a strong coreset guarantee in
practice.

All efficient coreset constructions are probabilistic and this comes with a disadvantage of coresets being difficult to evaluate. For example, it is co-NP-hard
to check whether a candidate compression is a weak coreset \cite{chrisESA}\footnote{A weak coreset guarantee only requires that a $(1+\varepsilon)$ approximation computed on
the coreset yields a $(1+\varepsilon)$ on the entire point set.}. Therefore, although coreset algorithms succeed with some high probability, it
is unclear how to computationally verify this. We refer to \cite{chrisESA} for further discussion on this topic and discuss our evaluation metrics in Section
\ref{sssec:metrics}.

\subsection{Quadtree Embeddings}

To design fast algorithms, one of the common techniques is to embed the Euclidean space into a tree metric using \emph{quadtree embeddings}.  The central idea
behind these is that each hypercube in the input space can be split into $2^d$ sub-cubes. We can represent this in a tree-structure, where each sub-cube has the
original hypercube as its parent. Appropriately setting the weight of each branch then preserves the expected distance between points in different sub-cubes
within an $O(\sqrt{d} \log \Delta)$ factor. Here, $\Delta$ is the \emph{spread} of the point set and is equal to the ratio of the largest distance over the
smallest non-zero distance.  Given this context, we now introduce quadtree embeddings more formally:

\paragraph*{Formal Overview}

Given a set of points in $\mathbb{R}^{n \times d}$, we want to return a tree-structure that roughly preserves their pairwise distances.  To do this, our first
step is to obtain a box enclosing all input points, centered at zero, with all side lengths equal to $\Delta$. This can be done as follows: select an arbitrary
input point, and translate the dataset so that this point is at the origin. Then, using $O(nd)$ time, set $\Delta$ to be the maximum distance from any point to
the origin. Note that, up to rescaling the points so that the smallest distance equals $1$, this is equivalent to the \emph{spread} as described in the previous
paragraph.

Having obtained this box, add a shift $s$ (picked uniformly at random in $[0, \Delta]$) to all the points' coordinates so that the input is now in the box
$[-2\Delta, 2\Delta]^d$. This transformation does not change any distances and therefore preserves the $k$-median and $k$-means costs.  The $i$-th level of the
tree (for $i \in \lbra 0, ..., \log \Delta \rbra$) is constructed by centering a grid of side length $2^{-i} \cdot 2\Delta$ at $0$, making each grid-cell a node
in the tree.  The parent of a cell $c$ is simply the cell at level $i-1$ that contains $c$, and the distance between $c$ and its parent is set to $2^{-i} \cdot
2\Delta \cdot \sqrt{d}$ (the length of the hypercube's diagonal). This embedding takes $O(nd \log \Delta)$ time to construct, where $\log \Delta$ is the depth
of the tree\footnote{Crucially, note that, in our \cref{alg:crudeApx}, we do not need to build the full tree.  Instead, we only require $\log \log \Delta$
levels of it, resulting in our improved complexity.}. The linearity in the $\log \Delta$ term comes from the fact that this is the maximum depth of the tree.

The distortion of this embedding is at most $O(d \log \Delta)$, as stated in the following lemma:
\begin{lemma}[Lemma 11.9 in \cite{har2011geometric}]\label{lem:quadtreeDist}
The distances in the tree metric $d_T$ satisfy
$\forall p,q, \|p-q\| \leq \mathbb{E}[d_T(p, q)] \leq O(d \log \Delta) \|p-q\|$, where the expectation is taken over the random shift $s$ of the decomposition.
\end{lemma}

A sample proof (and further intuition on quadtree embeddings) can be found in Lemma 11.9 in \cite{har2011geometric}. The result follows from combining linearity
of expectation and the fact that two points $p$ and $q$ are separated at level $i$ with probability at most $\sqrt{d} \|p-q\| \frac{2^i}{\Delta}$ (as in the
proof of \cref{lem:quadtreeSep}).
